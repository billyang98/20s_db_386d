\documentclass[12pt, letterpaper, fleqn]{article}
\usepackage[letterpaper, margin=.75in]{geometry}
\usepackage[utf8]{inputenc}
\usepackage{amsmath}
\usepackage{amssymb}
\usepackage{algorithmicx}
\usepackage{algpseudocode}
\usepackage{algorithm}
\usepackage[english]{babel}
\usepackage{amsthm}
\usepackage{graphicx}
\usepackage{xcolor}
\graphicspath{ {.} }
\usepackage{fancyhdr}
\usepackage{tikz}
\usepackage{hyperref}
\newcommand*\circled[1]{\tikz[baseline=(char.base)]{
            \node[shape=circle,draw,inner sep=2pt] (char) {#1};}}
\setlength\parindent{0pt}
\def\fullouterjoin{\mathbin{\ojoin\mkern-5.8mu\bowtie\mkern-5.8mu\ojoin}}

%\pagestyle{fancy}
%\fancyhf{}
%\rhead{Bill Yang}
%\renewcommand{\headrulewidth}{0pt}

\newcommand{\handout}[5]{
   \renewcommand{\thepage}{#1-\arabic{page}}
   \noindent
   \begin{center}
   \framebox{
      \vbox{
%    \hbox to 5.78in { {\bf M328K Number Theory} \hfill #2 }
%       \vspace{4mm}
%       \hbox to 5.78in { {\Large \hfill #5  \hfill} }
%       \vspace{2mm}
%       \hbox to 5.78in { {\it #3 \hfill #4} }
    \hbox to 5.78in { { Bill Yang} \hfill {Due: #2} }
       \vspace{4mm}
       \hbox to 5.78in { {\Large \hfill #5  \hfill} }
       \vspace{2mm}
       \hbox to 5.78in { {#3 \hfill #4} }
      }
   }
   \end{center}
   \vspace*{4mm}
}

\newcommand{\ho}[5]{\handout{#1}{#2}{#3}{Instructor: #4}{Homework #1}}

\begin{document}
  \ho{5b}{3/9/19}{CS386D Database Systems}{Daniel Miranker} \\

  \section{Part A}

  \textbf{15.3.2} \\
  $B(S) + (B(S)B(R)) / (M - 1) = 10000 + \lceil (10000/(1000-1)\rceil \times 10000) = \\
  120000$ disk I/O. \\
  
  \textbf{15.3.3 a} \\
  \begin{align*}
    100000 &= 10000 + \lceil 10000/ (M-1)\rceil (10000)  \\
    M &= 1112.11
  \end{align*}
  $M \geq 1113$ \\

  \textbf{16.2.6 a,b}\\
  a) \\
  $\pi_{b+c \rightarrow x, y}(\pi_{b,c}(R) \fullouterjoin \pi_{b, c, c+d
  \rightarrow y}(S))$ \\
  b) \\
  $\pi_{a,b,a+d \rightarrow z} (\pi_{a,b,c}(R) \fullouterjoin \pi_{b,c,d}(S))$

  \section{Part B}
  \textbf{1.} \\
  Let $c$ be the average seek time.\\
  Then the total cost would be $1000c$,
  needing to seek and read from all $B(R)$ blocks.\\
  
  \textbf{2.} \\
  Let $c$ be the average seek time and $c'$ be the average weighted rotational
  latency and track to track seek time. \\
  Then the cost would be $c + 999c'$, to
  seek the first block and then read the remaining blocks. \\

  \textbf{3.} \\
  a) \\
  Let $c$ be the average seek time. \\
  Then the cost would be $c$, since $W.c$ is
  the primary key, it is unique, and we only need to read \\
  one record. \\

  b) \\
  Since it is the primary key, I expect there to be\\
  $1$ record. \\

  \textbf{4.} \\
  a) \\
  Let $c$ be the average seek time. \\
  Then the cost would be $c$,\\
  since $(c,b,a)$ is
  a compound primary key, it is unique, and we only need to read one record,
  ignoring the index (B-tree) lookups. \\

  b) \\
  $1$ record since $(c,b,a)$ is a compound primary key.\\

  \textbf{5.} \\
  a) \\
  Let $c$ be the average seek time. There are $100$ values for attribute $a$ in
  the relation $R$, thus we expect there to be $100$ values. A block contains
  $1000$ values. Thus if sorted on primary key, all $100$ values should fall
  into the same block. \\ 
  Thus the total cost is $c$.\\

  b) \\
  Since there are $100$ values for attribute $a$, then we expect there to be
  \\
  $100$ records for the query.\\

  \textbf{6.} \\
  a) \\
  Since there are $20$ values of $b$, and $10$ values of $c$, if we assume that
  all combinations exist in the table, there are $200$ unique rows of $W$. Since
  there are $10000$ total rows, we can assume for a specific tuple $(b,c)$,
  there are $50$ rows that match. \\
  Thus we expect $50$ rows for this query.\\

  b) \\
  For $20$ values of $b$, we expect $500$ rows for a specific $b$. For a
  specific $c$, we can expect $1000$ rows.
  Thus the union, accounting for
  double-counting, would be $1500 - 50 =\\
  1450$ rows in this query. \\

  \textbf{7.} \\
  a) \\
  We assume $1/3$ of the rows satisfy the inequality. Then for a
  specific $c$ wold span $1/10$ of those rows, getting us $1/30$ of all rows, so
  \\
  $333$ rows expected.\\

  b) \\
  Similar to $7a$, we have $3333$ rows that meet the condition on $b$, and
  $1000$ rows that meet the condition on $c$. Thus accounting for double
  counting, we get \\
  $4000$ expected rows returned.\\

  \textbf{8.} \\
  $0$. No common attributes so no rows matched on a natural join.\\

  \textbf{9.} \\
  There are only $10$ values of $c$ in $W$, thus only $10/100$ of the rows of
  $V$ would get matched to a row in $W$. For a specific $c$ value in $W$, there
  are $1000$ rows of that value. Thus we have $100/10 = 10$ values of $c$ that
  are matched to $1000$ rows each in $W$. \\
  $10000$ rows returned. \\
  
  \textbf{10.} \\ 
  From $6a$, we saw that for a specific value of $(b,c)$, there are $50$ rows,
  and there are also only $200$ total values of $(b,c)$. In $R$, there are
  $10000$ total values of $(b,c)$. Thus for a only $200/10000$ of the total rows
  of $R$ would get matched in the join. $20000$ rows of $R$ joined to $50$ rows
  each of $W$ means \\
  $1000000$ total rows returned.\\

  \textbf{11.} \\
  $1/3$ of the rows in $R$ are matched by the inequality selection. $V(R,b) =
  200$ and $V(U,b) = 400$, so all values of $R.b$ are contained in the values of
  $U.b$.  For a single $b$ value in $U$, there are $25$ matching rows. Then
  we get \\
  $8333333$ rows returned.\\

  \textbf{12.} \\
  Let us first consider the $c$ join condition. There are $10$ values of $c$ in
  $W$ so for a given $c$, there are $1000$ rows. In $R$ there are $50$ values of
  $c$, so only $1/5$ of the rows in $R$ will be matched to something in $W$.
  Thus with $200000$ rows matched to $1000$ rows each, we get $200000000$ rows
  returned.\\
  Next we condition on $R.b > W.b$. Let us simplify further and consider $R.b >
  c$ for some constant $c$. Then $1/3$ of the rows would satisfy this constraint
  based on our simplifying assumptions. Then for some value of $W.b$, we
  consider the rows that are created after joining on $c$, $1/3$ of these rows
  should have $R.b > W.b$ based on our assumption.\\
  Thus we have $66666667$ rows returned.

\end{document}
