\documentclass[12pt, letterpaper, fleqn]{article}
\usepackage[letterpaper, margin=.75in]{geometry}
\usepackage[utf8]{inputenc}
\usepackage{amsmath}
\usepackage{amssymb}
\usepackage{algorithmicx}
\usepackage{algpseudocode}
\usepackage{algorithm}
\usepackage[english]{babel}
\usepackage{amsthm}
\usepackage{graphicx}
\usepackage{xcolor}
\graphicspath{ {.} }
\usepackage{fancyhdr}
\usepackage{tikz}
\usepackage{hyperref}
\newcommand*\circled[1]{\tikz[baseline=(char.base)]{
            \node[shape=circle,draw,inner sep=2pt] (char) {#1};}}
\setlength\parindent{0pt}

%\pagestyle{fancy}
%\fancyhf{}
%\rhead{Bill Yang}
%\renewcommand{\headrulewidth}{0pt}

\newcommand{\handout}[5]{
   \renewcommand{\thepage}{#1-\arabic{page}}
   \noindent
   \begin{center}
   \framebox{
      \vbox{
%    \hbox to 5.78in { {\bf M328K Number Theory} \hfill #2 }
%       \vspace{4mm}
%       \hbox to 5.78in { {\Large \hfill #5  \hfill} }
%       \vspace{2mm}
%       \hbox to 5.78in { {\it #3 \hfill #4} }
    \hbox to 5.78in { { Bill Yang} \hfill {Due: #2} }
       \vspace{4mm}
       \hbox to 5.78in { {\Large \hfill #5  \hfill} }
       \vspace{2mm}
       \hbox to 5.78in { {#3 \hfill #4} }
      }
   }
   \end{center}
   \vspace*{4mm}
}

\newcommand{\ho}[5]{\handout{#1}{#2}{#3}{Instructor: #4}{Homework #1}}

\begin{document}
  \ho{4}{2/28/19}{CS386D Database Systems}{Daniel Miranker} \\

  \textbf{1.}\\
  a) \\
  For each $c_i$, there are $m_i$ different values for the columns. Then for
  each column (using the result given in 14.7.3b), we need $2n \lceil \log_2 (m_i
  -1)\rceil$.
  Then the total number of bytes needed is\\
  $\frac{1}{8}\sum_{i=1}^{i=100} (2)100000000 \lceil \log_2 (m_i - 1)\rceil =
  25000000 \sum_{i=1}^{i=100} \lceil \log_2 (m_i - 1) \rceil $  \\

  b) \\
  No, it does not need one because each value of $c_0$ has only one row with
  that value which is recoverable from $c_0$, so it is redundant information.\\

  c) \\
  i. \\
  $50$ columns have $n/1000$ values and $50$ columns have $10,000$ values. Then
  for $S$, $n = 1000000$. For the $n/1000$ columns, we need $\frac{1}{8}50 (2) (1000000)
  \lceil \log_2(1000000/1000 - 1) \rceil = 125000000$ bytes.\\
  For the $10000$ columns, we need $\frac{1}{8}50 (2) (1000000)
  \lceil \log_2(10000 - 1) \rceil = 175000000$. \\
  Total we need \\
  $300000000$ bytes.\\

  ii. \\
  We need $\frac{1}{8}(50)(2)(100000000) (\lceil \log_2(100000000/1000 - 1)
  \rceil + \lceil \log_2(10000 - 1) \rceil) = $\\
  $38750000000$ bytes.\\
  
  d) \\
  i. \\
  For a single row, we need $4 + 50(25) + 50(20)$ bytes, since we need $4$ for
  $c_0$, then there are $50$ columns that require an average of $25$ bytes, and
  then other $50$ columns require an average of $20$ bytes.\\
  Thus for all $n$ rows, we need \\
  $2254n = 2254000000$ bytes. \\

  ii. \\
  $2254n = 225400000000$ bytes. \\

  iii. \\
  4k bytes = $2^{12} = 4096$ bytes.\\
  $2254000000 / 4096 \approx 550293$ pages.\\

  iv. \\
  $225400000000 / 4096 \approx 55029297$ pages.\\
  
  \textbf{2.}\\
  a) \\
  Then if there is a hit in the bloom fliter and the block is retrieved, we can
  verify whether the key actually exists in storage (the block). Otherwise we
  wouldn't know if we actually have the file needed or if it is a false
  positive.\\

  b) \\
  i. $64$Mbytes $= 2^{26}$ bytes. $1024 = 2^{10}$. $2^{26} / 2^{10} = 2^{16}$
  key value pairs fit in a block. Then the number of bits required for a filter
  is $(10) 2^{16} = $ \\
  $655360$ bits. \\

  ii. \\
  Database has $2^{43}$ total storage, across $128 = 2^7$ servers, so $2^{43} /
  2^{7} = 2^{36}$ storage per server. Then there are $2^{36} / 2^{26} = 2^{10}$
  blocks per server. Then the number of bytes needed for all the bloom filters is
  $10 (2^{16}) (2^{10}) / 8 = 10 (2^{23})= $\\
  $83886080$ bytes or $80 $Mbytes.\\

  iii. \\
  $\ln(2) \times m/n$, where $m/n = 10$. So the optimal number of hash functions is
  $7$. \\

  iv. \\
  Probability of false positive $= (1/2)^{\ln(2) \times m/n} = .00819$.
  



\end{document}
