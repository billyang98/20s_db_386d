\documentclass[12pt, letterpaper, fleqn]{article}
\usepackage[letterpaper, margin=.75in]{geometry}
\usepackage[utf8]{inputenc}
\usepackage{amsmath}
\usepackage{amssymb}
\usepackage{algorithmicx}
\usepackage{algpseudocode}
\usepackage{algorithm}
\usepackage[english]{babel}
\usepackage{amsthm}
\usepackage{graphicx}
\usepackage{xcolor}
\graphicspath{ {.} }
\usepackage{fancyhdr}
\usepackage{tikz}
\usepackage{hyperref}
\newcommand*\circled[1]{\tikz[baseline=(char.base)]{
            \node[shape=circle,draw,inner sep=2pt] (char) {#1};}}
\setlength\parindent{0pt}

%\pagestyle{fancy}
%\fancyhf{}
%\rhead{Bill Yang}
%\renewcommand{\headrulewidth}{0pt}

\newcommand{\handout}[5]{
   \renewcommand{\thepage}{#1-\arabic{page}}
   \noindent
   \begin{center}
   \framebox{
      \vbox{
%    \hbox to 5.78in { {\bf M328K Number Theory} \hfill #2 }
%       \vspace{4mm}
%       \hbox to 5.78in { {\Large \hfill #5  \hfill} }
%       \vspace{2mm}
%       \hbox to 5.78in { {\it #3 \hfill #4} }
    \hbox to 5.78in { { Bill Yang} \hfill {Due: #2} }
       \vspace{4mm}
       \hbox to 5.78in { {\Large \hfill #5  \hfill} }
       \vspace{2mm}
       \hbox to 5.78in { {#3 \hfill #4} }
      }
   }
   \end{center}
   \vspace*{4mm}
}

\newcommand{\ho}[5]{\handout{#1}{#2}{#3}{Instructor: #4}{Homework #1}}

\begin{document}
  \ho{3}{2/19/19}{CS386D Database Systems}{Daniel Miranker} \\

  \section{Part A}

  \textbf{14.7.1}\\
  \begin{center}
  \begin{tabular}{ |c | c| }
    \hline
    Speed        & Bitmap \\ \hline
    1.42         &  001000000000\\
    1.86         &  000000000010\\
    2.00         &  000000001000\\
    2.10         &  010000000000\\
    2.20         &  000000110000\\
    2.66         &  100000000000\\
    2.80         &  000100000101\\
    3.20         &  000011000000\\
    \hline
  \end{tabular}
  \end{center}
  \begin{center}
  \begin{tabular}{ |c | c| }
    \hline
    Speed        & Bitmap \\ \hline 
    1.42         &  1010\\
    1.86         &  11101011\\
    2.00         &  11101001\\
    2.10         &  01\\
    2.20         &  11011100\\
    2.66         &  00\\
    2.80         &  101111010101\\
    3.20         &  11010000\\
    \hline
  \end{tabular}
  \end{center}

  \begin{center}
  \begin{tabular}{ |c | c| }
    \hline
    Ram          & Bitmap \\ \hline
    512          &  011010000000\\
    1024         &  100101101001\\
    2048         &  000000010110\\
    \hline
  \end{tabular}
  \end{center}
  \begin{center}
  \begin{tabular}{ |c | c| }
    \hline
    Ram          & Bitmap \\ \hline
    512          &  010001\\
    1024         &  0010100100011010\\
    2048         &  1101110100\\
    \hline
  \end{tabular}
  \end{center}

  \begin{center}
  \begin{tabular}{ |c | c| }
    \hline
    HD           & Bitmap \\ \hline
    80           &  001000000000\\
    160          &  000000000011\\
    200          &  000000100000\\
    250          &  110110011000\\
    300          &  000000000100\\
    320          &  000001000000\\
    \hline
  \end{tabular}
  \end{center}
  \begin{center}
  \begin{tabular}{ |c | c| }
    \hline
    HD           & Bitmap \\ \hline
    80           &  1010\\
    160          &  1110101000\\
    200          &  110110\\
    250          &  00000100101000\\
    300          &  11101001\\
    320          &  110110\\
    \hline
  \end{tabular}
  \end{center}


  \textbf{14.7.3a}\\
  $\frac{1000000m}{8}$ bytes.
  $m$ bits per record. $1000000m$ total bits, $\frac{1000000m}{8}$ bytes.\\
  
  \section{Part B}
  \textbf{1.}\\
  a) I could go all the way to y without getting a false positive. \\ 
  b) I would expect the same results. \\

  \textbf{2.}\\
  a) A bit is $0$ with probability $(1 - \frac{1}{50})^3$, so the probability
  that we get a hit on all $3$ hashes is $(1 - (1 - \frac{1}{50})^3)^3 =
  .00020338$. \\
  b) A single bit is $0$ with probability $(1 - \frac{1}{50})^{3 * 20}$. The the
  probability that we get a false positive is $(1 - (1 - \frac{1}{50})^{3 *
  20})^3 =.3466 $\\
  c) False. Definitely not there.\\\\

  \textbf{3.}\\
  Given $m,n$, to minimize the probability of a false positive, we use $k = \ln
  2 \times m/n$. Since $m=50$ and $k=3$, we have $3 = \ln 2 \times 50/n$.
  Solving for $n$, we get $n = 50 /\frac{3}{\ln 2} = 11.55 \approx 12$ keys.

  \section{Part C}

  \subsection{1.}
  \textbf{5.1.1} \\
  $\pi_{speed}(PC) = $ \\
  \textit{SET} \\\\
  \begin{tabular} { |c| }
    \hline
    speed\\ \hline
    2.66 \\
    2.10 \\
    1.42 \\
    2.80 \\
    3.20 \\
    2.20 \\
    2.00 \\
    1.86 \\
    3.06  \\
    \hline
  \end{tabular} \\
  Average Value $= 2.36667$ \\\\

  \textit{BAG} \\\\
  \begin{tabular} { |c|}
    \hline
    speed\\ \hline
    2.66 \\
    2.10 \\
    1.42 \\
    2.80 \\
    3.20 \\
    3.20 \\
    2.20 \\
    2.20 \\
    2.00 \\
    2.80 \\
    1.86 \\
    2.80 \\
    3.06  \\
    \hline
  \end{tabular} \\
  Average Value $= 2.48462$ \\\\

  \textbf{5.1.2} \\
  $\pi_{hd}(PC) = $ \\
  \textit{SET}\\\\
  \begin{tabular} { | c|}
    \hline
    hd\\ \hline
    250 \\
    80 \\
    260 \\
    320 \\
    200 \\
    300 \\
    160 \\
    \hline
  \end{tabular} \\
  Average Value $= 224.29$ \\\\

  \textit{BAG} \\\\
  \begin{tabular} { |c|}
    \hline
    hd\\ \hline
    250 \\
    250 \\
    80 \\
    250 \\
    260 \\
    320 \\
    200 \\
    250 \\
    250 \\
    300 \\
    160 \\
    160 \\
    80  \\
    \hline
  \end{tabular} \\
  Average Value $= 216.15384$\\\\

  \textbf{16.2.2 b, c} \\
  b) \\
  Set difference\\\\
  \begin{tabular} {|c c|}
    \hline
    Relation R & \\ \hline
    Attribute A & Attribute B \\ \hline
    1  &  3 \\
    2  &  5 \\
    4  &  1 \\
    \hline
  \end{tabular}
  \begin{tabular} {| c c |}
    \hline
    Relation S & \\ \hline
    Attribute A & Attribute B \\  \hline
    1  &  2 \\
    2  &  5 \\
    5  &  2 \\ \hline
  \end{tabular} \\
  
  \begin{tabular} {| c |}
    \hline
    $\pi_A (R - S)$   \\ \hline
    Attribute A  \\ \hline
    1  \\
    4  \\ \hline
  \end{tabular} 

  \begin{tabular} {| c |}
    \hline
    $\pi_A (R) - \pi_A (S)$  \\ \hline
    Attribute A  \\ \hline
    4  \\ \hline
  \end{tabular} \\

  Bag difference\\\\
  \begin{tabular} {|c c|}
    \hline
    Relation R & \\ \hline
    Attribute A & Attribute B \\ \hline
    1  &  3 \\
    2  &  5 \\
    4  &  1 \\
    1  &  3 \\ 
    5  &  1 \\
    \hline
  \end{tabular}
  \begin{tabular} {| c c |}
    \hline
    Relation S & \\ \hline
    Attribute A & Attribute B \\  \hline
    1  &  2 \\
    2  &  5 \\
    5  &  2 \\
    1  &  3 \\
    1  &  2 \\ \hline
  \end{tabular} \\
  
  \begin{tabular} {| c |}
    \hline
    $\pi_A (R - S)$   \\ \hline
    Attribute A  \\ \hline
    4  \\
    1  \\
    5  \\ \hline
  \end{tabular} 

  \begin{tabular} {| c |}
    \hline
    $\pi_A (R) - \pi_A (S)$  \\ \hline
    Attribute A  \\ \hline
    4  \\ \hline
  \end{tabular} \\
  \\

  c) \\
  \begin{tabular} {|c c|}
    \hline
    Relation R & \\ \hline
    Attribute A & Attribute B \\ \hline
    1  &  3 \\
    1  &  5 \\
    1  &  3 \\
    \hline
  \end{tabular}
  
  \begin{tabular} {| c |}
    \hline
    $\delta (\pi_A (R) )$   \\ \hline
    Attribute A  \\ \hline
    1  \\ \hline
  \end{tabular} 

  \begin{tabular} {| c |}
    \hline
    $\pi_A (\delta(R) )$  \\ \hline
    Attribute A  \\ \hline
    1  \\ 
    1  \\ \hline
  \end{tabular} \\


  \subsection{2.}

  \textbf{i.}\\
  \begin{verbatim}
    select * from HW3_R where joinKey1 = 101;
  \end{verbatim}\\

  \begin{tabular}{ c c c }
    key & name & joinKey1 \\
    \hline 
    1	& Andrea	& 101
  \end{tabular} \\\\

  \textbf{ii.}\\
  \begin{verbatim}
    select * from HW3_R where joinKey1 != 101;
  \end{verbatim}\\

  \begin{tabular}{ c c c }
    key & name & joinKey1 \\
    \hline 
    5	& John	& 106
  \end{tabular} \\\\

  \textbf{iii.}\\
  \begin{verbatim}
    select * from HW3_R where joinKey1 is null;
  \end{verbatim}\\

  \begin{tabular}{ c c c }
    key & name & joinKey1 \\
    \hline 
    2 &	David &	NULL \\
    3	& David	& NULL \\
    4	& Dan	& NULL
  \end{tabular} \\\\

  \textbf{iv.}\\
  \begin{verbatim}
    select [name], joinKey1 from HW3_R;
  \end{verbatim}\\

  \begin{tabular}{ c c c }
    name & joinKey1 \\
    \hline 
    Andrea & 	101 \\
    David	& NULL \\
    David	& NULL \\
    Dan	& NULL \\
    John	& 106
  \end{tabular} \\\\

  \textbf{v.}\\
  \begin{verbatim}
    select joinKey1 from HW3_R;
  \end{verbatim}\\

  \begin{tabular}{ c c c }
    joinKey1 \\
    \hline 
    101 \\
    NULL \\
    NULL \\
    NULL \\
    106 \\
  \end{tabular} \\\\

  \textbf{vi.}\\
  \begin{verbatim}
    select * from HW3_R join HW3_S on joinKey1 = joinKey2;
  \end{verbatim}\\
  \begin{verbatim}
    select * from HW3_R cross join HW3_S where joinKey1 = joinKey2
  \end{verbatim}\\
  \begin{verbatim}
    select * from HW3_R full outer join HW3_S on joinKey1 = joinKey2 where
      joinKey1 = joinKey2
  \end{verbatim}\\

  \begin{tabular}{ c c c c c c }
    key & name & joinKey1 & key & romanNumeral & joinKey2 \\
    \hline 
    1	& Andrea & 101 &	6	& V	& 101
  \end{tabular} \\\\

  \textbf{vii.}\\
  \begin{verbatim}
    select * from HW3_R join HW3_S on joinKey1 != joinKey2
  \end{verbatim}\\
  \begin{verbatim}
    select * from HW3_R cross join HW3_S where joinKey1 != joinKey2
  \end{verbatim}\\
  \begin{verbatim}
    select * from HW3_R full outer join HW3_S on joinKey1 != joinKey2 where
      joinKey1 != joinKey2
  \end{verbatim}\\

  \begin{tabular}{ c c c c c c }
    key & name & joinKey1 & key & romanNumeral & joinKey2 \\
    \hline 
    5	& John	& 106	& 6&	V	& 101 \\
    1	& Andrea &	101	&8&	L	& 105 \\
    5	& John	& 106	& 8	& L	& 105 
  \end{tabular} \\\\

  \textbf{viii.}\\
  \begin{verbatim}
    select * from HW3_R full outer join HW3_S on joinKey1 = joinKey2
  \end{verbatim}\\
  \begin{verbatim}
    select * from HW3_R right outer join HW3_S on joinKey1 = joinKey2
    UNION
    select * from HW3_R left outer join HW3_S on joinKey1 = joinKey2
  \end{verbatim}\\

  \begin{tabular}{ c c c c c c }
    key & name & joinKey1 & key & romanNumeral & joinKey2 \\
    \hline 
    1	&Andrea	& 101	&6	&V	& 101 \\
    2	& David	& NULL	&NULL	&NULL&	NULL \\ 
    3	& David	& NULL	&NULL	&NULL	& NULL \\
    4	&Dan	& NULL	&NULL	& NULL	&NULL \\
    5	& John	& 106	&NULL	&NULL	& NULL \\
    NULL	& NULL &	NULL &	7	&X	&NULL \\
    NULL & 	NULL& 	NULL&	8	&L	&105 
  \end{tabular} \\\\

  \textbf{ix.}\\
  \begin{verbatim}
    select * from HW3_R left outer join HW3_S on joinKey1=joinKey2
  \end{verbatim}\\
  \begin{verbatim}
    select * from HW3_R full outer join HW3_S on joinKey1=joinKey2 where not
      (HW3_R.[key] is null and [name] is null and joinKey1 is null)
  \end{verbatim}\\

  \begin{tabular}{ c c c c c c }
    key & name & joinKey1 & key & romanNumeral & joinKey2 \\
    \hline 
    1	&Andrea	&101	&6	&V	&101 \\
    2	&David	&NULL	&NULL	&NULL	&NULL \\
    3	&David	&NULL	&NULL	&NULL	&NULL \\
    4	&Dan	&NULL	&NULL	&NULL	&NULL \\
    5	&John	&106	&NULL	&NULL	&NULL 
  \end{tabular} \\\\

  \textbf{x.}\\
  \begin{verbatim}
    select * from HW3_R where joinKey1 in (select joinKey2 from HW3_S)
  \end{verbatim}\\
  \begin{verbatim}
    select * from HW3_R where joinKey1 in (select joinKey2 from HW3_S where
    joinKey1 = joinKey2)
  \end{verbatim}\\

  \begin{tabular}{ c c c}
    key & name & joinKey1 \\
    \hline 
    1	&Andrea	&101
  \end{tabular} \\\\

  \textbf{xi.}\\
  \begin{verbatim}
    select * from HW3_R where not exists (select joinKey2 from HW3_S where
      joinKey1 = joinKey2)
  \end{verbatim}\\
  \begin{verbatim}
    select * from HW3_R where joinKey1 in (select joinKey1 from HW3_R except
      select joinKey2 from HW3_S) or joinKey1 is null
  \end{verbatim}\\

  \begin{tabular}{ c c c}
    key & name & joinKey1 \\
    \hline 
    2	&David	&NULL \\
    3	&David	&NULL \\
    4	&Dan	&NULL \\
    5	&John	&106
  \end{tabular} \\\\





\end{document}
